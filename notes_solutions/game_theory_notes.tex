% Notes and solutions

% -------------------------------------------------
% Package imports
% -------------------------------------------------
\documentclass[12pt, a4paper]{article}
\usepackage[utf8]{inputenc}% Input encoding
\usepackage[english]{babel}% Set language to english
\usepackage{graphicx}% For importing graphics
\usepackage{amsthm, amsfonts, amssymb, bm}% All the AMS packages
\usepackage{mathtools}% Fixes a few AMS bugs
\usepackage[expansion=false]{microtype}% Fixes to make typography better
\usepackage{hyperref}% For \href{URL}{text}
\usepackage{fancyhdr}% For fancy headers
\usepackage[sharp]{easylist}% Easy nested lists
\usepackage{parskip}% Web-like paragraphs
\usepackage{multicol}% For multiple columns
\usepackage{tikz-cd}% For diagrams
\usepackage{microtype}
\usepackage{listings}% To include source-code
\usepackage[margin = 2.5cm, includehead]{geometry}% May be used to set margins
\usepackage{nicefrac}% Enables \nicefrac{nom}{denom}
%\usepackage[sc]{mathpazo}% A nice font, alternative to CM
\usepackage{booktabs}
\usepackage{fancyvrb} % fancy verbatim
\usepackage{centernot} % For the NOT conditionally independent sign

\usepackage{accents}
\newcommand{\ubar}[1]{\underaccent{\bar}{#1}}

% -------------------------------------------------
% Package setup
% -------------------------------------------------

\newcommand{\Title}{\vspace*{-4em}Game theory and mechanism design}
\newcommand{\Author}{Christian \and Helge \and Jonas \and Tommy}
\newcommand{\listSpace}{-0.5em}% Global list space

\title{\Title}
\author{\Author}
\date{Last updated \today.}

% Shortcuts for sets and other stuff in mathematics
\newcommand{\Q}{\mathbb{Q}}
\newcommand{\R}{\mathbb{R}}
\newcommand{\D}{\mathcal{D}}
\newcommand{\F}{\mathcal{F}}
\newcommand{\Y}{\mathcal{Y}}
\newcommand{\Reg}{\mathcal{R}}
\newcommand{\Class}{\mathcal{C}}
\newcommand{\Z}{\mathbb{Z}}
\renewcommand{\sf}[1]{\mathsf{#1}}
\newcommand{\vect}[1]{\bm{#1}}
\newcommand{\norm}[1]{\left\lVert#1\right\rVert}
\newcommand{\abs}[1]{\left\lvert#1\right\rvert}

% \usepackage[]{natbib} 
% \citet{jon90} -> Jones et al.  (1990)
% \citep{jon90} -> (Jones et al., 1990)
% \citep[see][]{jon90} -> (see Jones et al., 1990)


% -------------------------------------------------
% Document start
% -------------------------------------------------
\begin{document}
\maketitle
\begin{abstract}
	\noindent 
	This document contains notes on game theory and mechanism design.
	
%		\citet{dixit_games_nodate}
%		
%				\citet{maschler_game_nodate}
%		
%		\citet{narahari_game_nodate}
%		
%		\citet{nisan_algorithmic_nodate}
%		
%		\citet{shoham_multiagent_nodate}
	

\end{abstract}

{\small \tableofcontents}


\clearpage

\section{Notes}

% ----------------------------------------------------------------------------
\subsection{Introduction and key notions}

Game theory is the study of how agents act under games.
Mechanism design the concerned with the design of the games themselves.
A rational agent chooses a strategy to maximize its utility.
An intelligent agent is able to compute its best strategy.

\begin{easylist}[itemize]
	\ListProperties(Space=\listSpace, Space*=\listSpace)
	# Some famous problems are
	## Student coordination
	## Battle of the sexes
	## Baess paradox
	## Prisoners dilemma
	## Sealed bid first price auction
	## Sealed bid second price auction (Vickrey auction)
	## Divide the dollar
	## Tradegy of the commons
	## Bandwith sharing game
	
	## Chicken
\end{easylist}


\begin{easylist}[itemize]
\ListProperties(Space=\listSpace, Space*=\listSpace)
# A strategic form (normal form) game is given by
\begin{equation*}
\Gamma = 
\langle N, (S_i)_{i \in N}, (u_i)_{i \in N}\rangle
\end{equation*}
where $N$ are the players, $S_i$ is the strategy set of player $i$ and $u_i$ is the utility function of player $i$.
The utility function maps from $S_1 \times \cdots S_n \to \mathbb{R}$.

# Players have a preference relation over the set of all strategy profiles $S$.
# Intelligence: each player is a game theorist.
# Common knowledge: every player knows it, every player knows that every player knows it, every player knows that every player knows that every player knows it, and so forth.
\end{easylist}

% ----------------------------------------------------------------------------
\subsection{Basic games and concepts}

\subsubsection*{Battle of the sexes}
\begin{table}[ht!]
	\centering
	\begin{tabular}{|c|c|c|} \hline
		& \multicolumn{2}{|c|}{2} \\ \hline
		1 & A & B \\ \hline
		A & $2, 1$ & $0,0$ \\ \hline
		B & $0,0$ & $1,2$ \\ \hline
	\end{tabular}
\end{table}

\begin{easylist}[itemize]
	\ListProperties(Space=\listSpace, Space*=\listSpace)
	# There is no dominant strategy equilibrium, since $(2, 0)$ and $(0, 1)$ are incomparable---neither dominates the other.
	# There are two PSNE: $(A, A)$ and $(B, B)$.
	# The MSNE is
	\begin{equation*}
		\sigma_1^* = ( 1/3, 2/3) \qquad
		\sigma_2^* = ( 2/3, 1/3)
	\end{equation*}
	and the expected utility for both players is $u_1(\sigma_1^*, \sigma_2^*) = u_2(\sigma_1^*, \sigma_2^*) = 2/3$.
	Notice that the expected utility for the MSNE is lower than either one of the PSNE.
\end{easylist}

\subsubsection*{Prisoners dilemma}
\begin{table}[ht!]
	\centering
	\begin{tabular}{|c|c|c|} \hline
		& \multicolumn{2}{|c|}{2} \\ \hline
		1 & A & B \\ \hline
		A & $-2, -2$ & $-8, -1$ \\ \hline
		B & $-1, -8$ & $-6, -6$ \\ \hline
	\end{tabular}
\end{table}

\begin{easylist}[itemize]
	\ListProperties(Space=\listSpace, Space*=\listSpace)
	# Dominant strategy equilibrium: the strategies $(B, B)$ are strongly dominant, since for player $1$: $B = (-1 ,-6) > A = (-2, -8)$.
	The same applies to player $2$.
	# Pure strategy Nash equilibrium: the strategies $(B, B)$ are a PSNE, since neither player will gain anything by unilaterally changing strategy.
	# The paradox is that $(A, A)$ strongly dominates $(B, B)$ for both players, but in $(A, A)$ an unilateral change of strategy would benefit both players.
	Therefore they both change to $B$ and both end up with less utility.
\end{easylist}

% ----------------------------------------------------------------------------
\subsection{Solution concepts}

\begin{easylist}[itemize]
	\ListProperties(Space=\listSpace, Space*=\listSpace)
	# Domination
	## Strong domination : Let $s_1$ and $s_2$ be strategies available to a player. The strategy $s_1$ strongly dominates $s_2$ if it yields the player higher utility no matter what the other players choose.
	### Example: $(4, 2, 1) > (3, 2, 0)$
	## Weak domination : Let $s_1$ and $s_2$ be strategies available to a player. The strategy $s_1$ weakly dominates $s_2$ if it yields the player at least as good utility no matter what the other players choose, and in at least one case a better utility.
	### Example: $(4, 2, 1) \geq (4, 1, 1)$
	## Very weak domination : Let $s_1$ and $s_2$ be strategies available to a player. The strategy $s_1$ very weakly dominates $s_2$ if it yields the player no worse utility no matter what the other players choose.
	### Example: $(4, 1, 1) \geqq (4, 1, 1)$
	# A strategy profile $(s_1^*, \ldots, s_n^*)$ for all players is a (strong / weak / very weak) strategy equilibrium if strategy $s_i^*$ is a a (strong / weak / very weak) strategy for all players $i =1, \ldots, n$.
	
	# Pure strategy Nash equilibrium (PSNE)
	## A strategy profile $(s_1^*, \ldots, s_n^*)$ is a PSNE if no player gains anything by unilaterally switching strategy.
	In other words, for each player $i$ we must have
	\begin{equation*}
		u_i\left( s_i^*, s_{-i}^* \right)
		\geq 
		u_i\left( s_i, s_{-i}^* \right)
		\quad
		\forall s_i \in S_i.
	\end{equation*}
	A game may have no PSNE, one, PSNE or several PSNE.
	## Every dominant strategy equilibrium is a PSNE
	\begin{equation}
		\text{Strong Eq.}
		\subset
		\text{Weak Eq.}
		\subset
		\text{Very Weak Eq.}
		\subset
		\text{PSNE}
	\end{equation}
	## Interpretations:
	### Prescription given by an external advisor to the agents
	### Possible prediction given rationality and intelligence of agents
	### Self enforcing agreement where no agent has incentive to deviate
	### Convergence point of plays
	
	# Maxmin and Minmax values and strategies
	## Consider the following game, where the utilities are for player 1:
	\begin{table}[ht!]
		\centering
		\begin{tabular}{|c|c|c|c|} \hline
			& \multicolumn{3}{|c|}{2} \\ \hline
			1 & A & B & C \\ \hline
			A & 5 & 4 & 3  \\ \hline
			B & 2 & 7 & 8 \\ \hline
			C & 1 & 4 & 6 \\ \hline
		\end{tabular}
	\end{table}
	## \textbf{Maxmin strategy of player 1.} If Player 1 goes first, he can choose $A$ to guarantee a utility of $3$.
	\begin{align*}
	\ubar{v} &= \max_{s_{i}} \min_{s_{-i}}  u_i
	\left( s_i, s_{-i} \right)
	\\
	&=
	\max_{s_{i}}
	\left\{ 
	\min_{s_{-i}} \left\{ 5, 9, 3 \right\},
	\min_{s_{-i}} \left\{ 2, 7, 8 \right\},
	\min_{s_{-i}} \left\{ 1, 4, 6 \right\}
	\right\} \\
	&=
	\max_{s_{i}}
	\left\{ 
	3, 2, 1
	\right\} = 3
	\end{align*}
	
	## \textbf{Minmax strategy of player 1.} If player 2 goes first, Player 1 can $A$ to guarantee a utility of $5$.
	\begin{align*}
	\bar{v} &= \min_{s_{-i}} \max_{s_{i}}   u_i
	\left( s_i, s_{-i} \right)
	\\
	&=
	\min_{s_{-i}}
	\left\{ 
	\max_{s_{i}} \left\{ 5, 2, 1 \right\},
	\max_{s_{i}}  \left\{ 9, 7, 4 \right\},
	\max_{s_{i}}  \left\{ 3, 8, 6 \right\}
	\right\} \\
	&=
	\min_{s_{-i}}
	\left\{ 
	5, 9, 8
	\right\} = 5
	\end{align*}
	## A PSNE for is no less than than the minmax strategy, which is in turn no less than the maxmin strategy.
	\begin{equation*}
		u_i \left( s_i^*, s_{-i}^* \right)
		\geq
		\bar{v}_i
		\geq
		\ubar{v}_i
	\end{equation*}
	
\end{easylist}




\clearpage
\section{Exercises}

% ----------------------------------------------------------------------------
\subsubsection*{Exercise 6.1}
\textbf{Show in a strategic form game that any strongly (weakly) (very weakly) dominant strategy equilibrium is also a pure strategy Nash equilibrium.}

Let $s^* = (s_1^*, ..., s_n^*)$ be a dominant strategy equilibrium and $i\in [1,...,n]$ be an arbitrary player. $s_i^*$ is a dominant strategy for $i$, such that
\[ u_i(s_i^*, s_{-i}) \geq u_i(s_i,s_{-i})\ \forall s_{-i}\in S_{-i} \]

Hence
\[ u_i(s_i^*, s_{-i}^*) \geq u_i(s_i, s_{-i}^*) \]
and $s^*$ must also be a Nash equilibrium.

% ----------------------------------------------------------------------------
\subsubsection*{Exercise 6.3}

\textbf{Find the pure strategy Nash equilibria, maxmin values, minmax values, maxmin strategies, and minmax strategies of the following game.}

\begin{table}[ht!]
	\centering
	\begin{tabular}{|c|c|c|} \hline
		& \multicolumn{2}{|c|}{2} \\ \hline
		1 & A & B \\ \hline
		A & 0,1 & 1,1 \\ \hline
		B & 1,1 & 1,0 \\ \hline
	\end{tabular}
\end{table}

\textit{pure strategy Nash equilibrium} $(A,B)$, $(B,A)$

\textit{maxmin values} $\underline{v_1}=1$, $\underline{v_2}=1$

\textit{maxmin strategies} $s_1=\{B\}$, $s_2=\{A\}$

\textit{minmax values} $\overline{v_1}=1$, $\overline{v_2}=1$

\textit{minmax strategies} $s_1=\{A, B\}$, $s_2=\{A,B\}$


% ----------------------------------------------------------------------------
\subsubsection*{Exercise 6.9}

\textbf{Give examples of two player pure strategy games for the following situations}

\begin{enumerate}
	\item[(a)] \textbf{The game has a unique Nash equilibrium which is not a weakly dominant strategy equilibrium}
	
	\begin{table}[ht!]
		\centering
		\begin{tabular}{|c|c|c|} \hline
			& \multicolumn{2}{|c|}{2} \\ \hline
			1 & A & B \\ \hline
			A & 1,0 & 0,1 \\ \hline
			B & 0,1 & 0,0 \\ \hline
		\end{tabular}
	\end{table}
	
	$(A,B)$ is a unique Nash equilibrium.
	
	\item[(b)] \textbf{The game has a unique Nash equilibrium which is a weakly dominant strategy equilibrium but not a strongly dominant strategy equilibrium}
	
	\begin{table}[ht!]
		\centering
		\begin{tabular}{|c|c|c|} \hline
			& \multicolumn{2}{|c|}{2} \\ \hline
			1 & A & B \\ \hline
			A & 1,1 & 0,0 \\ \hline
			B & 0,1 & 0,0 \\ \hline
		\end{tabular}
	\end{table}
	
	$(A,A)$ is a unique Nash equilibrium and a weakly dominant strategy equilibrium.
	
	\item[(c)] \textbf{The game has one strongly dominant or one weakly dominant strategy equilibrium and a second one which is only a Nash equilibrium}
	
	\begin{table}[ht!]
		\centering
		\begin{tabular}{|c|c|c|} \hline
			& \multicolumn{2}{|c|}{2} \\ \hline
			1 & A & B \\ \hline
			A & 1,1 & 0,1 \\ \hline
			B & 0,1 & 0,0 \\ \hline
		\end{tabular}
	\end{table}
	
	$(A,A)$ is a weakly dominant strategy equilibrium and $(A,B)$ is only a Nash equilibrium.
	
\end{enumerate}

% ----------------------------------------------------------------------------
\subsubsection*{Exercise 6.10}

\textbf{Assume two bidders with valuations $v_1$ and $v_2$ for an object. Their bids are in multiples of some unit (that is, discrete). The bidder with higher bid wins the auction and pays the amount that he has bid. If both bid the same amount, one of them gets the object with equal probability $\frac{1}{2}$. In this game, compute a pure strategy Nash equilibrium of the game.}

There are three possible strategies; a bidder $i$ may bid over his own valuation $b_i > v_i$, equal to his valuation $b_i = v_i$ or under his valuation $b_i < v_i$.

Depending on the bid of the other bidders $j$, the utility for $i$ is given as follows
\[
u_i =
\begin{cases}
v_i - b_i & \text{ if } b_i > b_j \\
\frac{1}{2}(v_i - b_i) & \text{ if } b_i = b_j \\
0 & \text{ if } b_i < b_j \\
\end{cases}
\]

There are nine possible outcomes for $i$:
\begin{table}[ht!]
	\centering
	\begin{tabular}{|c|c|c|c|} \hline
		& $b_i > b_j$ & $b_i = b_j$ & $b_i < b_j$ \\ \hline
		$b_i > v_i$ & $<0$ & $<0$ & $0$ \\ \hline
		$b_i = v_i$ & $0$ & $0$ & $0$ \\ \hline
		$b_i < v_i$ & $>0$ & $>0$ & $0$ \\ \hline
	\end{tabular}
\end{table}

Given that $i$ is an arbitrary bidder, the payoff matrix is the same for all bidders, hence $b_i < v_i$ must be a pure strategy Nash equilibrium.


% ----------------------------------------------------------------------------
\subsubsection*{Exercise 7.1}

\textbf{Let $S$ be any finite set with $n$ elements. Show that the set $\Delta(S)$, the set of all probability distributions over $S$, is a convex set.}


% ----------------------------------------------------------------------------
\subsubsection*{Exercise 7.5}
% TODO: Equations are wrong (results are correct)

\textbf{Find the mixed strategy Nash equilibria for the rock-paper-scissors game:}

\begin{table}[ht!]
    \centering
    \begin{tabular}{|c|c|c|c|}                   \hline
                 & \multicolumn{3}{|c|}{2}    \\ \hline
        1        & Rock   & Paper  & Scissors \\ \hline
        Rock     & $0,0$  & $-1,1$ & $1,-1$   \\ \hline
        Paper    & $1,-1$ & $0,0$  & $-1,1$   \\ \hline
        Scissors & $-1,1$ & $1,-1$ & $0,0$    \\ \hline
    \end{tabular}
\end{table}

\textbf{Also compute the maxmin value and minmax value in mixed strategies. Determine the maxmin mixed strategies of each player and the minmax mixed strategies against each player.}

Player 1 plays Rock with probability $x$, Paper with probability $y$ and Scissors with probability $1-x-y$.

Player 1 is indifferent between Rock and Paper if 
\[ -y + (1-x-y) = x - (1-x-y) \Leftrightarrow y = \frac{2}{3} - x \]
and indifferent between Paper and Scissors if
\[ x - (1-x-y) = -x + y \Leftrightarrow  x = \frac{1}{3}\]

Player 2 plays Rock with probability $p$, Paper with probability $q$ and Scissors with probability $1-p-q$.

Player 2 is indifferent between Rock and Paper if
\[ -q + (1-p-q) = p - (1-p-q) \Leftrightarrow q = \frac{2}{3} - p \]
and indifferent between Paper and Scissors if
\[ p - (1-p-q) = -p + q \Leftrightarrow p = \frac{1}{3} \]

This results in the following mixed strategy Nash equilibrium:
\[ \sigma_1^* = \left(\frac{1}{3}, \frac{1}{3}, \frac{1}{3}\right)\quad
\sigma_2^* = \left(\frac{1}{3}, \frac{1}{3}, \frac{1}{3}\right) \]


% ----------------------------------------------------------------------------
\subsubsection*{Exercise 7.8}
% TODO: Equations are wrong (results are reversed)

\textbf{Find the mixed strategy Nash equilibria for the following game.}

\begin{table}[ht!]
    \centering
    \begin{tabular}{|c|c|c|} \hline
          & A     & B     \\ \hline
        A & $6,2$ & $0,0$ \\ \hline
        B & $0,0$ & $2,6$ \\ \hline
    \end{tabular}
\end{table}

\[ 6x = 2(1-x) \Leftrightarrow x = \frac{1}{4} \]
\[ 2p = 6(1-p) \Leftrightarrow p = \frac{3}{4} \]

Mixed strategy Nash equilibrium is given by $\sigma_1^* = \left(\frac{1}{4}, \frac{3}{4}\right)$ and $\sigma_2^* = \left(\frac{3}{4}, \frac{1}{4}\right)$.

\textbf{If all these numbers are multiplied by 2, will the equilibria change?}

Multiplying all numbers by 2, results in the exact same equilibrium:

\[ 12x = 4(1-x) \Leftrightarrow x = \frac{1}{4} \]
\[ 4p = 12(1-p) \Leftrightarrow p = \frac{3}{4} \]


% ----------------------------------------------------------------------------
\subsubsection*{Exercise 7.10}

\textbf{This game is called the \emph{guess the average} game. There are $n$ players. Each player announces a number in the set $\{1, ..., K\}$. A monetary reward of \$1 is split equally between all the players whose number is closest to $\frac{2}{3}$ of the average number. Formulate this as a strategic form game. Show that the game has a unique mixed strategy Nash equilibrium, in which each player plays a pure strategy.}

Each player has $K$ strategies, where the strategy $s_i$ for $i=1, ..., K$ represents announcing the number $i$. The mixed strategy Nash equilibrium is found by iterated elimination of weakly dominated strategies. The first iteration eliminates all strategies $s_j$ for $j = \frac{2}{3}K, ..., K$, because none of these strategies can possible be $\frac{2}{3}$ of the average number. This continues until all but one strategy have been eliminated, namely $s_1$.


% ----------------------------------------------------------------------------
\subsubsection*{Exercise 8.4}

\textbf{Prove Theorem 8.2 which provides a convenient characterization for risk neutral,
risk averse, and risk loving players.}


% ----------------------------------------------------------------------------
\subsubsection*{Exercise 8.2}

\textbf{Complete the proof of the result that affine transformations of a utility function do not affect properties (1) and (2) of the von Neumann – Morgenstern utilities (see Theorem 8.1).}


% ----------------------------------------------------------------------------
\subsubsection*{Exercise 8.1}

\textbf{Complete the proof of Lemma 8.1.}


% ----------------------------------------------------------------------------
\subsubsection*{Exercise 9.3}

\textbf{An $m \times m$ matrix is called a latin square if each row and each column is a permutation of $(1, . . . , m)$. Compute pure strategy Nash equilibria, if they exist, of a matrix game for which a latin square is the payoff matrix.}

Example of a latin square for $m=3$:

\begin{equation*}
    A = 
    \begin{bmatrix}
        1 & 3 & 2 \\
        2 & 1 & 3 \\
        3 & 2 & 1 \\
    \end{bmatrix}
\end{equation*}

Here, we have:

\[ \underline{v} = \max_i \min_j a_{ij} = \max\{1, 1, 1\} = 1 \]
\[ \overline{v} = \min_j \max_i a_{ij} = \min\{3, 3, 3\} = 3 \]

Since $\underline{v} \neq \overline{v}$, the matrix game has no pure strategy Nash equilibrium.

Note that $\underline{v}$ is always $1$, since all rows always contain $1$, and that $\overline{v}$ is always $m$ since all rows contain $m$. Hence, the latin square has a pure strategy Nash equilibrium if and only if $m=1$.


% ----------------------------------------------------------------------------
\subsubsection*{Exercise 9.7}

\textbf{Give an example of a matrix game for each of the following cases:}

\begin{itemize}
    \item \textbf{There exist only pure strategy Nash equilibria}
    
    \[A_1 = \begin{bmatrix}
        2 & 3 \\
        1 & 2 \\
    \end{bmatrix}\]

    $(A, A)$ is the only pure strategy Nash equilibrium.

    \item \textbf{There exists exactly one Nash equilibrium}
    
    \[A_2 = \begin{bmatrix}
        1 & -1 \\
        -1 & 1 \\
    \end{bmatrix}\]

    Exactly one Nash equilibrium at $(\frac{1}{2}A, \frac{1}{2}B)$.
    
    \item \textbf{There exist exactly two Nash equilibria}
    
    \[A_3 = \begin{bmatrix}
        1 & 0 \\
        1 & 1 \\
    \end{bmatrix}\]

    Two Nash equilibria at $(B, A)$ and $(B, B)$.
    
    \item \textbf{There exist infinite number of Nash equilibria}
    
    \[A_4 = \begin{bmatrix}
        1 & 1 \\
        1 & 1 \\
    \end{bmatrix}\]
    
    There are infinitely many mixed Nash equilibria in this game.
    
    \item \textbf{There exists a strongly dominant strategy equilibrium}
    
    \[A_5 = \begin{bmatrix}
        2 & 3 \\
        1 & 2 \\
    \end{bmatrix}\]

    $(A, A)$ is a strongly dominant strategy equilibrium.

\end{itemize}


% ----------------------------------------------------------------------------
\subsubsection*{Exercise 9.9}

\textbf{For the following matrix game}

    \[A = \begin{bmatrix}
        2 & 3 & 1 \\
        4 & 1 & 2 \\
        4 & 1 & 3
    \end{bmatrix}\]

\begin{itemize}
    \item \textbf{Compute maxmin and minmax values over pure strategies}
    
    \[ \underline{v} = \max_i \min_j a_{ij} = \max\{1, 1, 1\} = 1 \]
    \[ \overline{v} = \min_j \max_i a_{ij} = \min\{4, 3, 3\} = 3 \]


    \item \textbf{Compute all pure strategy Nash equilibria}
    
    No pure strategy nash equilibria

    \item \textbf{Compute maxmin and minmax values over mixed strategies}
    
    Linear programs computed at https://online-optimizer.appspot.com/.

    \begin{verbatim}
var x1 >= 0;
var x2 >= 0;
var x3 >= 0;
var z;

maximize obj: z;

subject to c0: x1 + x2 + x3 = 1;
subject to c1: z <= 2*x1 + 4*x2 + 4*x3;
subject to c2: z <= 3*x1 + x2 + x3;
subject to c3: z <= x1 + 2*x2 + 3*x3;

end;
    \end{verbatim}

    Optimal solution $z=2$ for mixed strategy $\left(\frac{1}{2}, 0, \frac{1}{2}\right)$.

    \begin{verbatim}
var y1 >= 0;
var y2 >= 0;
var y3 >= 0;
var w;

minimize obj: w;

subject to c0: y1 + y2 + y3 = 1;
subject to c1: w >= 2*y1 + 3*y2 + y3;
subject to c2: w >= 4*y1 + y2 + 2*y3;
subject to c3: w >= 4*y1 + y2 + 3*y3;

end;
    \end{verbatim}

    Optimal solution $w=2$ for mixed strategy $\left(0, \frac{1}{2}, \frac{1}{2}\right)$.

    \item \textbf{Compute all mixed strategy Nash equilibria}
    
    A mixed strategy Nash equilibrium is $\left(\left(\frac{1}{2}, 0, \frac{1}{2}\right), \left(0, \frac{1}{2}, \frac{1}{2} \right) \right)$

\end{itemize}



\bibliographystyle{apalike}%alpha, apalike is also good
\bibliography{bibliography}
\end{document}
