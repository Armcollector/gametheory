% Notes and solutions

% -------------------------------------------------
% Package imports
% -------------------------------------------------
\documentclass[12pt, a4paper]{article}
\usepackage[utf8]{inputenc}% Input encoding
\usepackage[english]{babel}% Set language to english
\usepackage{graphicx}% For importing graphics
\usepackage{amsthm, amsfonts, amssymb, bm}% All the AMS packages
\usepackage{mathtools}% Fixes a few AMS bugs
\usepackage[expansion=false]{microtype}% Fixes to make typography better
\usepackage{hyperref}% For \href{URL}{text}
\usepackage{fancyhdr}% For fancy headers
\usepackage[sharp]{easylist}% Easy nested lists
\usepackage{parskip}% Web-like paragraphs
\usepackage{multicol}% For multiple columns
\usepackage{tikz-cd}% For diagrams
\usepackage{microtype}
\usepackage{listings}% To include source-code
\usepackage[margin = 2.5cm, includehead]{geometry}% May be used to set margins
\usepackage{nicefrac}% Enables \nicefrac{nom}{denom}
%\usepackage[sc]{mathpazo}% A nice font, alternative to CM
\usepackage{booktabs}
\usepackage{fancyvrb} % fancy verbatim
\usepackage{centernot} % For the NOT conditionally independent sign

\usepackage{accents}
\newcommand{\ubar}[1]{\underaccent{\bar}{#1}}

% -------------------------------------------------
% Package setup
% -------------------------------------------------

\newcommand{\Title}{\vspace*{-4em}Game theory and mechanism design}
\newcommand{\Author}{Christian \and Helge \and Jonas \and Tommy}
\newcommand{\listSpace}{-0.5em}% Global list space

\title{\Title}
\author{\Author}
\date{Last updated \today.}

% Shortcuts for sets and other stuff in mathematics
\newcommand{\Q}{\mathbb{Q}}
\newcommand{\R}{\mathbb{R}}
\newcommand{\D}{\mathcal{D}}
\newcommand{\F}{\mathcal{F}}
\newcommand{\Y}{\mathcal{Y}}
\newcommand{\Reg}{\mathcal{R}}
\newcommand{\Class}{\mathcal{C}}
\newcommand{\Z}{\mathbb{Z}}
\renewcommand{\sf}[1]{\mathsf{#1}}
\newcommand{\vect}[1]{\bm{#1}}
\newcommand{\norm}[1]{\left\lVert#1\right\rVert}
\newcommand{\abs}[1]{\left\lvert#1\right\rvert}

% \usepackage[]{natbib} 
% \citet{jon90} -> Jones et al.  (1990)
% \citep{jon90} -> (Jones et al., 1990)
% \citep[see][]{jon90} -> (see Jones et al., 1990)


% -------------------------------------------------
% Document start
% -------------------------------------------------
\begin{document}
\maketitle
\begin{abstract}
	\noindent 
	This document contains notes on game theory and mechanism design.
	
%		\citet{dixit_games_nodate}
%		
%				\citet{maschler_game_nodate}
%		
%		\citet{narahari_game_nodate}
%		
%		\citet{nisan_algorithmic_nodate}
%		
%		\citet{shoham_multiagent_nodate}
	

\end{abstract}

{\small \tableofcontents}


\clearpage

\section{Notes}

% ----------------------------------------------------------------------------
\subsection{Introduction and key notions}

Game theory is the study of how agents act under games.
Mechanism design the concerned with the design of the games themselves.
A rational agent chooses a strategy to maximize its utility.
An intelligent agent is able to compute its best strategy.

\begin{easylist}[itemize]
	\ListProperties(Space=\listSpace, Space*=\listSpace)
	# Some famous problems are
	## Student coordination
	## Battle of the sexes
	## Baess paradox
	## Prisoners dilemma
	## Sealed bid first price auction
	## Sealed bid second price auction (Vickrey auction)
	## Divide the dollar
	## Tradegy of the commons
	## Bandwith sharing game
	
	## Chicken
\end{easylist}


\begin{easylist}[itemize]
\ListProperties(Space=\listSpace, Space*=\listSpace)
# A strategic form (normal form) game is given by
\begin{equation*}
\Gamma = 
\langle N, (S_i)_{i \in N}, (u_i)_{i \in N}\rangle
\end{equation*}
where $N$ are the players, $S_i$ is the strategy set of player $i$ and $u_i$ is the utility function of player $i$.
The utility function maps from $S_1 \times \cdots S_n \to \mathbb{R}$.

# Players have a preference relation over the set of all strategy profiles $S$.
# Intelligence: each player is a game theorist.
# Common knowledge: every player knows it, every player knows that every player knows it, every player knows that every player knows that every player knows it, and so forth.
\end{easylist}

% ----------------------------------------------------------------------------
\subsection{Basic games and concepts}

\subsubsection*{Battle of the sexes}
\begin{table}[ht!]
	\centering
	\begin{tabular}{|c|c|c|} \hline
		& \multicolumn{2}{|c|}{2} \\ \hline
		1 & A & B \\ \hline
		A & $2, 1$ & $0,0$ \\ \hline
		B & $0,0$ & $1,2$ \\ \hline
	\end{tabular}
\end{table}

\begin{easylist}[itemize]
	\ListProperties(Space=\listSpace, Space*=\listSpace)
	# There is no dominant strategy equilibrium, since $(2, 0)$ and $(0, 1)$ are incomparable---neither dominates the other.
	# There are two PSNE: $(A, A)$ and $(B, B)$.
	# The MSNE is
	\begin{equation*}
		\sigma_1^* = ( 1/3, 2/3) \qquad
		\sigma_2^* = ( 2/3, 1/3)
	\end{equation*}
	and the expected utility for both players is $u_1(\sigma_1^*, \sigma_2^*) = u_2(\sigma_1^*, \sigma_2^*) = 2/3$.
	Notice that the expected utility for the MSNE is lower than either one of the PSNE.
\end{easylist}

\subsubsection*{Prisoners dilemma}
\begin{table}[ht!]
	\centering
	\begin{tabular}{|c|c|c|} \hline
		& \multicolumn{2}{|c|}{2} \\ \hline
		1 & A & B \\ \hline
		A & $-2, -2$ & $-8, -1$ \\ \hline
		B & $-1, -8$ & $-6, -6$ \\ \hline
	\end{tabular}
\end{table}

\begin{easylist}[itemize]
	\ListProperties(Space=\listSpace, Space*=\listSpace)
	# Dominant strategy equilibrium: the strategies $(B, B)$ are strongly dominant, since for player $1$: $B = (-1 ,-6) > A = (-2, -8)$.
	The same applies to player $2$.
	# Pure strategy Nash equilibrium: the strategies $(B, B)$ are a PSNE, since neither player will gain anything by unilaterally changing strategy.
	# The paradox is that $(A, A)$ strongly dominates $(B, B)$ for both players, but in $(A, A)$ an unilateral change of strategy would benefit both players.
	Therefore they both change to $B$ and both end up with less utility.
\end{easylist}

% ----------------------------------------------------------------------------
\subsection{Solution concepts}

\begin{easylist}[itemize]
	\ListProperties(Space=\listSpace, Space*=\listSpace)
	# Domination
	## Strong domination : Let $s_1$ and $s_2$ be strategies available to a player. The strategy $s_1$ strongly dominates $s_2$ if it yields the player higher utility no matter what the other players choose.
	### Example: $(4, 2, 1) > (3, 2, 0)$
	## Weak domination : Let $s_1$ and $s_2$ be strategies available to a player. The strategy $s_1$ weakly dominates $s_2$ if it yields the player at least as good utility no matter what the other players choose, and in at least one case a better utility.
	### Example: $(4, 2, 1) \geq (4, 1, 1)$
	## Very weak domination : Let $s_1$ and $s_2$ be strategies available to a player. The strategy $s_1$ very weakly dominates $s_2$ if it yields the player no worse utility no matter what the other players choose.
	### Example: $(4, 1, 1) \geqq (4, 1, 1)$
	# A strategy profile $(s_1^*, \ldots, s_n^*)$ for all players is a (strong / weak / very weak) strategy equilibrium if strategy $s_i^*$ is a a (strong / weak / very weak) strategy for all players $i =1, \ldots, n$.
	
	# Pure strategy Nash equilibrium (PSNE)
	## A strategy profile $(s_1^*, \ldots, s_n^*)$ is a PSNE if no player gains anything by unilaterally switching strategy.
	In other words, for each player $i$ we must have
	\begin{equation*}
		u_i\left( s_i^*, s_{-i}^* \right)
		\geq 
		u_i\left( s_i, s_{-i}^* \right)
		\quad
		\forall s_i \in S_i.
	\end{equation*}
	A game may have no PSNE, one, PSNE or several PSNE.
	## Every dominant strategy equilibrium is a PSNE
	\begin{equation}
		\text{Strong Eq.}
		\subset
		\text{Weak Eq.}
		\subset
		\text{Very Weak Eq.}
		\subset
		\text{PSNE}
	\end{equation}
	## Interpretations:
	### Prescription given by an external advisor to the agents
	### Possible prediction given rationality and intelligence of agents
	### Self enforcing agreement where no agent has incentive to deviate
	### Convergence point of plays
	
	# Maxmin and Minmax values and strategies
	## Consider the following game, where the utilities are for player 1:
	\begin{table}[ht!]
		\centering
		\begin{tabular}{|c|c|c|c|} \hline
			& \multicolumn{3}{|c|}{2} \\ \hline
			1 & A & B & C \\ \hline
			A & 5 & 4 & 3  \\ \hline
			B & 2 & 7 & 8 \\ \hline
			C & 1 & 4 & 6 \\ \hline
		\end{tabular}
	\end{table}
	## \textbf{Maxmin strategy of player 1.} If Player 1 goes first, he can choose $A$ to guarantee a utility of $3$.
	\begin{align*}
	\ubar{v} &= \max_{s_{i}} \min_{s_{-i}}  u_i
	\left( s_i, s_{-i} \right)
	\\
	&=
	\max_{s_{i}}
	\left\{ 
	\min_{s_{-i}} \left\{ 5, 9, 3 \right\},
	\min_{s_{-i}} \left\{ 2, 7, 8 \right\},
	\min_{s_{-i}} \left\{ 1, 4, 6 \right\}
	\right\} \\
	&=
	\max_{s_{i}}
	\left\{ 
	3, 2, 1
	\right\} = 3
	\end{align*}
	
	## \textbf{Minmax strategy of player 1.} If player 2 goes first, Player 1 can $A$ to guarantee a utility of $5$.
	\begin{align*}
	\bar{v} &= \min_{s_{-i}} \max_{s_{i}}   u_i
	\left( s_i, s_{-i} \right)
	\\
	&=
	\min_{s_{-i}}
	\left\{ 
	\max_{s_{i}} \left\{ 5, 2, 1 \right\},
	\max_{s_{i}}  \left\{ 9, 7, 4 \right\},
	\max_{s_{i}}  \left\{ 3, 8, 6 \right\}
	\right\} \\
	&=
	\min_{s_{-i}}
	\left\{ 
	5, 9, 8
	\right\} = 5
	\end{align*}
	## A PSNE for is no less than than the minmax strategy, which is in turn no less than the maxmin strategy.
	\begin{equation*}
		u_i \left( s_i^*, s_{-i}^* \right)
		\geq
		\bar{v}_i
		\geq
		\ubar{v}_i
	\end{equation*}
	
\end{easylist}




\clearpage
\section{Exercises}

% ----------------------------------------------------------------------------
\subsubsection*{Exercise 6.1}
\textbf{Show in a strategic form game that any strongly (weakly) (very weakly) dominant strategy equilibrium is also a pure strategy Nash equilibrium.}

Let $s^* = (s_1^*, ..., s_n^*)$ be a dominant strategy equilibrium and $i\in [1,...,n]$ be an arbitrary player. $s_i^*$ is a dominant strategy for $i$, such that
\[ u_i(s_i^*, s_{-i}) \geq u_i(s_i,s_{-i})\ \forall s_{-i}\in S_{-i} \]

Hence
\[ u_i(s_i^*, s_{-i}^*) \geq u_i(s_i, s_{-i}^*) \]
and $s^*$ must also be a Nash equilibrium.

% ----------------------------------------------------------------------------
\subsubsection*{Exercise 6.3}

\textbf{Find the pure strategy Nash equilibria, maxmin values, minmax values, maxmin strategies, and minmax strategies of the following game.}

\begin{table}[ht!]
	\centering
	\begin{tabular}{|c|c|c|} \hline
		& \multicolumn{2}{|c|}{2} \\ \hline
		1 & A & B \\ \hline
		A & 0,1 & 1,1 \\ \hline
		B & 1,1 & 1,0 \\ \hline
	\end{tabular}
\end{table}

\textit{pure strategy Nash equilibrium} $(A,B)$, $(B,A)$

\textit{maxmin values} $\underline{v_1}=1$, $\underline{v_2}=1$

\textit{maxmin strategies} $s_1=\{B\}$, $s_2=\{A\}$

\textit{minmax values} $\overline{v_1}=1$, $\overline{v_2}=1$

\textit{minmax strategies} $s_1=\{A, B\}$, $s_2=\{A,B\}$


% ----------------------------------------------------------------------------
\subsubsection*{Exercise 6.9}

\textbf{Give examples of two player pure strategy games for the following situations}

\begin{enumerate}
	\item[(a)] \textbf{The game has a unique Nash equilibrium which is not a weakly dominant strategy equilibrium}
	
	\begin{table}[ht!]
		\centering
		\begin{tabular}{|c|c|c|} \hline
			& \multicolumn{2}{|c|}{2} \\ \hline
			1 & A & B \\ \hline
			A & 1,0 & 0,1 \\ \hline
			B & 0,1 & 0,0 \\ \hline
		\end{tabular}
	\end{table}
	
	$(A,B)$ is a unique Nash equilibrium.
	
	\item[(b)] \textbf{The game has a unique Nash equilibrium which is a weakly dominant strategy equilibrium but not a strongly dominant strategy equilibrium}
	
	\begin{table}[ht!]
		\centering
		\begin{tabular}{|c|c|c|} \hline
			& \multicolumn{2}{|c|}{2} \\ \hline
			1 & A & B \\ \hline
			A & 1,1 & 0,0 \\ \hline
			B & 0,1 & 0,0 \\ \hline
		\end{tabular}
	\end{table}
	
	$(A,A)$ is a unique Nash equilibrium and a weakly dominant strategy equilibrium.
	
	\item[(c)] \textbf{The game has one strongly dominant or one weakly dominant strategy equilibrium and a second one which is only a Nash equilibrium}
	
	\begin{table}[ht!]
		\centering
		\begin{tabular}{|c|c|c|} \hline
			& \multicolumn{2}{|c|}{2} \\ \hline
			1 & A & B \\ \hline
			A & 1,1 & 0,1 \\ \hline
			B & 0,1 & 0,0 \\ \hline
		\end{tabular}
	\end{table}
	
	$(A,A)$ is a weakly dominant strategy equilibrium and $(A,B)$ is only a Nash equilibrium.
	
\end{enumerate}

% ----------------------------------------------------------------------------
\subsubsection*{Exercise 6.10}

\textbf{Assume two bidders with valuations $v_1$ and $v_2$ for an object. Their bids are in multiples of some unit (that is, discrete). The bidder with higher bid wins the auction and pays the amount that he has bid. If both bid the same amount, one of them gets the object with equal probability $\frac{1}{2}$. In this game, compute a pure strategy Nash equilibrium of the game.}

There are three possible strategies; a bidder $i$ may bid over his own valuation $b_i > v_i$, equal to his valuation $b_i = v_i$ or under his valuation $b_i < v_i$.

Depending on the bid of the other bidders $j$, the utility for $i$ is given as follows
\[
u_i =
\begin{cases}
v_i - b_i & \text{ if } b_i > b_j \\
\frac{1}{2}(v_i - b_i) & \text{ if } b_i = b_j \\
0 & \text{ if } b_i < b_j \\
\end{cases}
\]

There are nine possible outcomes for $i$:
\begin{table}[ht!]
	\centering
	\begin{tabular}{|c|c|c|c|} \hline
		& $b_i > b_j$ & $b_i = b_j$ & $b_i < b_j$ \\ \hline
		$b_i > v_i$ & $<0$ & $<0$ & $0$ \\ \hline
		$b_i = v_i$ & $0$ & $0$ & $0$ \\ \hline
		$b_i < v_i$ & $>0$ & $>0$ & $0$ \\ \hline
	\end{tabular}
\end{table}

Given that $i$ is an arbitrary bidder, the payoff matrix is the same for all bidders, hence $b_i < v_i$ must be a pure strategy Nash equilibrium.


% ----------------------------------------------------------------------------
\subsubsection*{Exercise 7.1}

\textbf{Let $S$ be any finite set with $n$ elements. Show that the set $\Delta(S)$, the set of all probability distributions over $S$, is a convex set.}


% ----------------------------------------------------------------------------
\subsubsection*{Exercise 7.5}
% TODO: Equations are wrong (results are correct)

\textbf{Find the mixed strategy Nash equilibria for the rock-paper-scissors game:}

\begin{table}[ht!]
    \centering
    \begin{tabular}{|c|c|c|c|}                   \hline
                 & \multicolumn{3}{|c|}{2}    \\ \hline
        1        & Rock   & Paper  & Scissors \\ \hline
        Rock     & $0,0$  & $-1,1$ & $1,-1$   \\ \hline
        Paper    & $1,-1$ & $0,0$  & $-1,1$   \\ \hline
        Scissors & $-1,1$ & $1,-1$ & $0,0$    \\ \hline
    \end{tabular}
\end{table}

\textbf{Also compute the maxmin value and minmax value in mixed strategies. Determine the maxmin mixed strategies of each player and the minmax mixed strategies against each player.}

Player 1 plays Rock with probability $x$, Paper with probability $y$ and Scissors with probability $1-x-y$.

Player 1 is indifferent between Rock and Paper if 
\[ -y + (1-x-y) = x - (1-x-y) \Leftrightarrow y = \frac{2}{3} - x \]
and indifferent between Paper and Scissors if
\[ x - (1-x-y) = -x + y \Leftrightarrow  x = \frac{1}{3}\]

Player 2 plays Rock with probability $p$, Paper with probability $q$ and Scissors with probability $1-p-q$.

Player 2 is indifferent between Rock and Paper if
\[ -q + (1-p-q) = p - (1-p-q) \Leftrightarrow q = \frac{2}{3} - p \]
and indifferent between Paper and Scissors if
\[ p - (1-p-q) = -p + q \Leftrightarrow p = \frac{1}{3} \]

This results in the following mixed strategy Nash equilibrium:
\[ \sigma_1^* = \left(\frac{1}{3}, \frac{1}{3}, \frac{1}{3}\right)\quad
\sigma_2^* = \left(\frac{1}{3}, \frac{1}{3}, \frac{1}{3}\right) \]


% ----------------------------------------------------------------------------
\subsubsection*{Exercise 7.8}
% TODO: Equations are wrong (results are reversed)

\textbf{Find the mixed strategy Nash equilibria for the following game.}

\begin{table}[ht!]
    \centering
    \begin{tabular}{|c|c|c|} \hline
          & A     & B     \\ \hline
        A & $6,2$ & $0,0$ \\ \hline
        B & $0,0$ & $2,6$ \\ \hline
    \end{tabular}
\end{table}

\[ 6x = 2(1-x) \Leftrightarrow x = \frac{1}{4} \]
\[ 2p = 6(1-p) \Leftrightarrow p = \frac{3}{4} \]

Mixed strategy Nash equilibrium is given by $\sigma_1^* = \left(\frac{1}{4}, \frac{3}{4}\right)$ and $\sigma_2^* = \left(\frac{3}{4}, \frac{1}{4}\right)$.

\textbf{If all these numbers are multiplied by 2, will the equilibria change?}

Multiplying all numbers by 2, results in the exact same equilibrium:

\[ 12x = 4(1-x) \Leftrightarrow x = \frac{1}{4} \]
\[ 4p = 12(1-p) \Leftrightarrow p = \frac{3}{4} \]


% ----------------------------------------------------------------------------
\subsubsection*{Exercise 7.10}

\textbf{This game is called the \emph{guess the average} game. There are $n$ players. Each player announces a number in the set $\{1, ..., K\}$. A monetary reward of \$1 is split equally between all the players whose number is closest to $\frac{2}{3}$ of the average number. Formulate this as a strategic form game. Show that the game has a unique mixed strategy Nash equilibrium, in which each player plays a pure strategy.}

Each player has $K$ strategies, where the strategy $s_i$ for $i=1, ..., K$ represents announcing the number $i$. The mixed strategy Nash equilibrium is found by iterated elimination of weakly dominated strategies. The first iteration eliminates all strategies $s_j$ for $j = \frac{2}{3}K, ..., K$, because none of these strategies can possible be $\frac{2}{3}$ of the average number. This continues until all but one strategy have been eliminated, namely $s_1$.


% ----------------------------------------------------------------------------
\subsubsection*{Exercise 8.4}

\textbf{Prove Theorem 8.2 which provides a convenient characterization for risk neutral,
risk averse, and risk loving players.}


% ----------------------------------------------------------------------------
\subsubsection*{Exercise 8.2}

\textbf{Complete the proof of the result that affine transformations of a utility function do not affect properties (1) and (2) of the von Neumann – Morgenstern utilities (see Theorem 8.1).}


% ----------------------------------------------------------------------------
\subsubsection*{Exercise 8.1}

\textbf{Complete the proof of Lemma 8.1.}


% ----------------------------------------------------------------------------
\subsubsection*{Exercise 9.3}

\textbf{An $m \times m$ matrix is called a latin square if each row and each column is a permutation of $(1, . . . , m)$. Compute pure strategy Nash equilibria, if they exist, of a matrix game for which a latin square is the payoff matrix.}

Example of a latin square for $m=3$:

\begin{equation*}
    A = 
    \begin{bmatrix}
        1 & 3 & 2 \\
        2 & 1 & 3 \\
        3 & 2 & 1 \\
    \end{bmatrix}
\end{equation*}

Here, we have:

\[ \underline{v} = \max_i \min_j a_{ij} = \max\{1, 1, 1\} = 1 \]
\[ \overline{v} = \min_j \max_i a_{ij} = \min\{3, 3, 3\} = 3 \]

Since $\underline{v} \neq \overline{v}$, the matrix game has no pure strategy Nash equilibrium.

Note that $\underline{v}$ is always $1$, since all rows always contain $1$, and that $\overline{v}$ is always $m$ since all rows contain $m$. Hence, the latin square has a pure strategy Nash equilibrium if and only if $m=1$.


% ----------------------------------------------------------------------------
\subsubsection*{Exercise 9.7}

\textbf{Give an example of a matrix game for each of the following cases:}

\begin{itemize}
    \item \textbf{There exist only pure strategy Nash equilibria}
    
    \[A_1 = \begin{bmatrix}
        2 & 3 \\
        1 & 2 \\
    \end{bmatrix}\]

    $(A, A)$ is the only pure strategy Nash equilibrium.

    \item \textbf{There exists exactly one Nash equilibrium}
    
    \[A_2 = \begin{bmatrix}
        1 & -1 \\
        -1 & 1 \\
    \end{bmatrix}\]

    Exactly one Nash equilibrium at $(\frac{1}{2}A, \frac{1}{2}B)$.
    
    \item \textbf{There exist exactly two Nash equilibria}
    
    \[A_3 = \begin{bmatrix}
        1 & 0 \\
        1 & 1 \\
    \end{bmatrix}\]

    Two Nash equilibria at $(B, A)$ and $(B, B)$.
    
    \item \textbf{There exist infinite number of Nash equilibria}
    
    \[A_4 = \begin{bmatrix}
        1 & 1 \\
        1 & 1 \\
    \end{bmatrix}\]
    
    There are infinitely many mixed Nash equilibria in this game.
    
    \item \textbf{There exists a strongly dominant strategy equilibrium}
    
    \[A_5 = \begin{bmatrix}
        2 & 3 \\
        1 & 2 \\
    \end{bmatrix}\]

    $(A, A)$ is a strongly dominant strategy equilibrium.

\end{itemize}


% ----------------------------------------------------------------------------
\subsubsection*{Exercise 9.9}

\textbf{For the following matrix game}

    \[A = \begin{bmatrix}
        2 & 3 & 1 \\
        4 & 1 & 2 \\
        4 & 1 & 3
    \end{bmatrix}\]

\begin{itemize}
    \item \textbf{Compute maxmin and minmax values over pure strategies}
    
    \[ \underline{v} = \max_i \min_j a_{ij} = \max\{1, 1, 1\} = 1 \]
    \[ \overline{v} = \min_j \max_i a_{ij} = \min\{4, 3, 3\} = 3 \]


    \item \textbf{Compute all pure strategy Nash equilibria}
    
    No pure strategy nash equilibria

    \item \textbf{Compute maxmin and minmax values over mixed strategies}
    
    Linear programs computed at https://online-optimizer.appspot.com/.

    \begin{verbatim}
var x1 >= 0;
var x2 >= 0;
var x3 >= 0;
var z;

maximize obj: z;

subject to c0: x1 + x2 + x3 = 1;
subject to c1: z <= 2*x1 + 4*x2 + 4*x3;
subject to c2: z <= 3*x1 + x2 + x3;
subject to c3: z <= x1 + 2*x2 + 3*x3;

end;
    \end{verbatim}

    Optimal solution $z=2$ for mixed strategy $\left(\frac{1}{2}, 0, \frac{1}{2}\right)$.

    \begin{verbatim}
var y1 >= 0;
var y2 >= 0;
var y3 >= 0;
var w;

minimize obj: w;

subject to c0: y1 + y2 + y3 = 1;
subject to c1: w >= 2*y1 + 3*y2 + y3;
subject to c2: w >= 4*y1 + y2 + 2*y3;
subject to c3: w >= 4*y1 + y2 + 3*y3;

end;
    \end{verbatim}

    Optimal solution $w=2$ for mixed strategy $\left(0, \frac{1}{2}, \frac{1}{2}\right)$.

    \item \textbf{Compute all mixed strategy Nash equilibria}
    
    A mixed strategy Nash equilibrium is $\left(\left(\frac{1}{2}, 0, \frac{1}{2}\right), \left(0, \frac{1}{2}, \frac{1}{2} \right) \right)$

\end{itemize}


% ----------------------------------------------------------------------------
\subsubsection*{Exercise 13.1}

\textbf{Write down the definitions of weakly dominant strategy equilibrium and strongly dominant strategy equilibrium for Bayesian games.}

Given a Bayesian game
\[ \Gamma = \langle N,(\theta_i),(S_i),(p_i),(u_i)\rangle \]
a profile of strategies $(s_i^*, ..., s_n^*)$ is called a strongly dominant strategy equilibrium if $\forall i\in N; \forall s_i : \Theta_i \rightarrow S_i; \forall s_{-i} : \Theta_{-i} \rightarrow S_{-i}; \forall \theta_i \in \Theta_i$
\[u_i((s_i^*, s_{-i})|\theta_i) > u_i((s_i, s_{-i})|\theta_i)\]


% ----------------------------------------------------------------------------
\subsubsection*{Exercise 13.4}

\textbf{Consider two agents 1 and 2 where agent 1 is the seller of an indivisible item and agent 2 is a prospective buyer of the item. The type $\theta_1$ of agent 1 (seller) can be interpreted as the willingness to sell of the agent (minimum price at which agent 1 is willing to sell). The type $\theta_2$ of agent 2 (buyer) has the natural interpretation of willingness to pay (maximum price the buyer is willing to pay). Assume that $\theta_1 = \theta_2 = [0, 1]$ and that each agent thinks that the type of the other agent is uniformly distributed over the real interval $[0, 1]$. Define the following protocol. The seller and the buyer are asked to submit their bids $b_1$ and $b_2$ respectively. Trade happens if $b_1 \leq b_2$ and trade does not happen otherwise. If trade happens, the buyer gets the item and pays the seller an amount $\frac{(b_1+b_2)}{2}$. Compute a Bayesian Nash equilibrium of the Bayesian game here.}

Assume the bid of the buyer (agent 2) takes the form $b_2(\theta_2) = \alpha\theta_2$, where $\alpha \in (0,1]$. The seller wishes to maximize his expected payoff $\max(b_1 - \theta_1)P\{b_1 \leq b_2(\theta_2)\}$. Since the bid of agent 2 is $b_2(\theta_2)=\alpha\theta_2$ and $\theta_2 \in [0,1]$, the maximum bid of agent 2 is $\alpha$. The seller (agent 1) knows this and therefore $b_1 \in [0, \alpha]$. Futhermore,

\begin{align*}
	P\{b_1 \leq b_2(\theta_2)\} &= P\{b_1 \leq \alpha\theta_2\} \\
	&= P\{\frac{b_1}{\alpha} \leq \theta_2\} \\
	&= 1 - \frac{b_1}{\alpha}
\end{align*}

Hence the seller (agent 1) wants to solve 
\[\max_{b_1\in[0,\alpha]}(b_1 - \theta_1)\left(1 - \frac{b_1}{\alpha}\right)\]
such that
\[ b_1(\theta_1) = \frac{\theta_1 + \alpha}{2}\]

% ----------------------------------------------------------------------------
\subsubsection*{Exercise 14.1}

\textbf{In the cake cutting problem, if there were three children rather than two, what would be some appropriate mechanisms for fair sharing of the cake?}

\begin{enumerate}
	\item The first child cuts a slice of the cake.
	\item The second child has two options:
	\begin{enumerate}
		\item Choose the slice cut by the first child.
		\item Slice the remaining cake.
	\end{enumerate}
	\item Depending on the action of the previous child, the next child can either:
	\begin{enumerate}
		\item Choose between any of the cut slices.
		\item Slice the remaining cake.
	\end{enumerate}
	\item Perform at most $n-1$ cuts and continue untill all slices have been distributed.
\end{enumerate}

% ----------------------------------------------------------------------------
\subsubsection*{Exercise 14.3}

\textbf{Write down the type sets, outcomes, and social choice function in the baby’s mother problem.}

\begin{itemize}
	\item \textbf{Type sets}
	Each agent is either the baby's mother or not.
	\item \textbf{Outcomes}
	Each agent can either plead with the king or not.
	\item \textbf{Social choice function}
	An agent will plead if and only if the agent is the baby's mother.
\end{itemize}


% ----------------------------------------------------------------------------
\subsubsection*{Exercise 14.4}

\textbf{Write down the social choice functions SCF1 and SCF2 assuming one selling agent and $n$ buying agents.}

Rewriting SCF1 assuming $n$ buying agents, we have for $i = (1,...,n)$:

\begin{align*}
	y_0(\theta) &= 0 \\
	y_i(\theta) &=
		\begin{cases}
			1 & \text{ if } \theta_i = \max(\theta)	\\
			0 & \text{ otherwise}
		\end{cases} \\
	t_i(\theta) &= -y_i(\theta) \theta_i \\
	t_0(\theta) &= -\sum_i t_i(\theta)
\end{align*}

and for SCF2, the only difference is $t_i$ which in this case is:

\begin{equation*}
	t_i = -y_i(\theta) \theta_k \text{ for } k=\text{argmax}(\theta_{-i})
\end{equation*}


% ----------------------------------------------------------------------------
\subsubsection*{Exercise 15.4}

\textbf{Show that the social choice function BUY1 cannot be implemented by a direct mechanism.}

The social choice function BUY1 is SCF1 modified to represent the buying scenario.

It is defined as
\begin{align*}
	y_0(\theta) &= 0 \\
	y_1(\theta) &= 1 \text{ if } \theta_1 \leq \theta_2 \\
				&= 0 \text{ if } \theta_1 > \theta_2 \\
	y_2(\theta) &= 1 \text{ if } \theta_1 > \theta_2 \\
				&= 0 \text{ if } \theta_1 \leq \theta_2 \\
	t_1(\theta) &= y_1(\theta)\theta_1 \\
	t_2(\theta) &= y_2(\theta)\theta_2 \\
	t_0(\theta) &= -(t_1(\theta) + t_2(\theta)) \\
\end{align*}
such that the buyer buys the object from the seller with the lowest willingness to sell.

To show that BUY1 cannot be implemented by a direct mechansim, assume that seller 2 announces his true value $\theta_2$. Seller 1 wishes to maximize his expected payoff, having valuation $\theta_1$ and by announcing $\hat{\theta}_1$. If $\hat{\theta}_1 \leq \theta_2$, then seller 1 wins and his utility will be $\hat{\theta}_1 - \theta_1$. If $\hat{\theta}_1 > \theta_2$, then seller 2 wins and seller 1's utility is zero. Hence, seller 1 wishes to solve the problem
\[ \max_{\hat{\theta}_1}(\hat{\theta}_1 - \theta_1)P\{\hat{\theta}_1 \leq \theta_2\} \]
where
\[ P\{\hat{\theta}_1 \leq \theta_2\} = 1-\hat{\theta}_1 \]

The problem then becomes
\[ \max_{\hat{\theta}_1}(\hat{\theta}_1 - \theta_1)(1-\hat{\theta}_1) \]
hence
\[ \hat{\theta}_1 = \frac{\theta_2}{2} + \frac{1}{2} \]
which shows that seller 1 is incentivized to lie about his true type.

% ----------------------------------------------------------------------------
\subsubsection*{Exercise 15.8}

\textbf{In some of the examples discussed in this chapter, it is assumed that a tie between agent 1 and agent 2 will be resolved in favor of agent 1. Investigate what would happen in these examples if the tie is resolved in favor of agent 1 or agent 2 using a Bernoulli random variable.}

First of all, social functions must take the tie into account. Secondly, the utility maximization problems become dependant on the Bernoulli random variable. This could also affect the implementability of SCFs.

% ----------------------------------------------------------------------------
\subsubsection*{Exercise 15.9}

\textbf{See page 235-236 in the textbook.}

The utility for the seller is defined as
\[ u_1 = \frac{b_1 + b_2}{2} - b_1 \]
and for the buyer as
\[ u_2 = b_2 - \frac{b_1 + b_2}{2} \]



% ----------------------------------------------------------------------------
\subsubsection*{Exercise 16.1}

\textbf{Consider the cake-cutting problem with 3 children instead of 2. For this problem, suggest (i) a mechanism which will implement the SCF in dominant strategies and (ii) a mechanism which will implement the SCF in Bayesian Nash equilibrium but not in dominant strategies. Prove your results with simple, brief, logical arguments.}

\begin{enumerate}
	\item[i)] The solution to exercise 14.1 above implements the SCF in dominant strategies, as each child is responisble for cutting his own slice of cake. This is because the last to cut the cake is also the last to pick a slice of cake, such that if a child cuts a smaller slice, he is left with that slice himself.
	\item[ii)] A mechanism where each child cuts a slice of cake for the next child implements the SCF in Bayesian Nash equilibrium.
\end{enumerate}

% ----------------------------------------------------------------------------
\subsubsection*{Exercise 16.2}

\textbf{Show that the social choice functions SCF1 and SCF2 we defined in Chapter 15 are ex-post efficient. How about SCF3?}

% ----------------------------------------------------------------------------
\subsubsection*{Exercise 16.10}

% ----------------------------------------------------------------------------
\subsubsection*{Exercise 18.1}

\textbf{Consider five selling agents $\{1, 2, 3, 4, 5\}$, with valuations $v_1 = 20; v_2 = 15; v_3 = 12; v_4 = 10; v_5 = 6$, participating in a sealed bid procurement auction for a single indivisible item. These valuations are to be viewed as the willingness to sell values of the bidders.}
\begin{enumerate}
	\item[a)] \textbf{If Vickrey auction is the mechanism used, then compute the allocation and payment.}
	 
	The dominant strategy for the agents is to bid their valuations.
	Agent 5 is the winner with a marginal contribution of $6 - 10 = -4$, hence the agent receives $6 - (-4)= 10$.

	\item[b)] \textbf{If the buyer wishes to procure three objects instead of one object, and each bidder can supply at most one object, then compute the allocation and payments.}
	
	Agent 5, 4 and 3 are the winners with marginal contributions
	\[ (6 + 10 + 12) - (10 + 12 + 15) = -9 \]
	\[ (6 + 10 + 12) - ( 6 + 12 + 15) = -5 \]
	\[ (6 + 10 + 12) - ( 6 + 10 + 15) = -3 \]
	Hence, agent 5, 4 and 3 receive $6 - (-9) = 15$, $10 - (-5) = 15$ and $12 - (-3) = 15$.

	\item[c)] \textbf{Finally, if the buyer wishes to buy 6 objects and each bidder is willing to supply up to two objects, compute the allocation and payments.}
	
	Agent 5, 4 and 3 are the winners with marginal contributions
	\[ (6 + 6 + 10 + 10 + 12 + 12) - (10 + 10 + 12 + 12 + 15 + 15) = -18 \]
	\[ (6 + 6 + 10 + 10 + 12 + 12) - ( 6 +  6 + 12 + 12 + 15 + 15) = -10 \]
	\[ (6 + 6 + 10 + 10 + 12 + 12) - ( 6 +  6 + 10 + 10 + 15 + 15) = -6 \]
	Hence, agent 5, 4 and 3 receive $6 + 6 - (-18) = 30$, $10 + 10 - (-10) = 30$ and $12 + 12 - (-6) = 30$.

\end{enumerate}

% ----------------------------------------------------------------------------
\subsubsection*{Exercise 18.2}

\textbf{Consider an exchange where a single unit of an item is traded. There are 4 sellers S1, S2, S3, S4 and 3 buyers B1, B2, and B3. Here are the bids from the buyers and the asks from the sellers for the single item. The objective is to maximize the surplus in the exchange.}
\begin{verbatim}
S1: 10 S2: 12 S3: 14 S4: 16
B1: 8  B2: 12 B3: 18
\end{verbatim}
\begin{enumerate}
	\item[a)] \textbf{Define the surplus of the exchange as the total amount of money paid by the buyers minus the total amount of money received by the sellers. Call an allocation surplus maximizing if it maximizes the surplus of the exchange. Find a surplus maximizing allocation for this exchange.}

	Let $y_S=(s_1,s_2,s_3,s_4)$ be the sellers' allocation vector where $s_1 = 1$ if seller S1 sells his item and 0 otherwise. Similarly, let $y_B=(b_1, b_2, b_3)$ be the buyers' allocation vector where $b_1 = 1$ if buyer B1 buys an item. $K$ is the set of $(y_S, y_B)$ pairs which satisfy $\sum y_S = \sum y_B$.

	The surplus is defined as $y_B\cdot(B1, B2, B3) - y_S\cdot (S1, S2, S3, S4)$.

	A surplus maximizing allocation is $((1, 0, 0, 0), (0, 0, 1))$ with a surplus of $18 - 10 = 8$.

	\item[b)] \textbf{Assuming that Vickrey pricing is used, what will be the payment?}
	
	Buyer 3 has a marginal contribution of $18 - 12 = 6$, such that his payment will be $18 - 6 = 12$.

	\item[c)] \textbf{Will the mechanism satisfy budget balance?}
	
	The mechanism satisfies budget balance since the system is closed and all payments made by any buyer are received by a seller.

\end{enumerate}
% ----------------------------------------------------------------------------
\subsubsection*{Exercise 18.3}

\textbf{Consider a forward auction for sale of $m$ identical objects. Let there be $n$ bidders where $n > m$. The valuations of the bidders for the object are $v_1, ..., v_n$, respectively. Each bidder is interested in at most one unit.}

\begin{enumerate}
	\item[a)] \textbf{For this auction scenario, write down an allocation rule that is allocatively efficient.}
	Let $y=(y_1,...,y_n)$ be the allocation vector where $y_1 = 1$ is bidder 1 wins an item, and 0 otherwise. $K$ is the set of $y$ where $\sum y = m$.

	An allocation rule that is allocatively efficient is to select the $m$ bidders with largest valuations.

	\item[b)] \textbf{What will	be the Clarke payment in this case?}
	
	The Clarke payment is given by
	\[ h_i = -\sum_{j\neq i} v_j\cdot y_j\]
	
	\item[c)] \textbf{Do you see any difficulty? How can you overcome the difficulty, if any?}


\end{enumerate}

% ----------------------------------------------------------------------------
\subsubsection*{Exercise 18.4}

\textbf{Consider an auction for selling a single indivisible item where the bidder with the highest bid is declared as the winner and the winner pays an amount equal to twice the bid of the bidder with the lowest valuation among the rest of the agents. Is this mechanism a Clarke mechanism? Is the mechanism a Groves mechanism? Is the mechanism incentive compatible?}


\bibliographystyle{apalike}%alpha, apalike is also good
\bibliography{bibliography}
\end{document}
